\chapter*{Abstract}

Health care faces the problem of dealing with a vast amount of heterogenous data, that isn't easily accessible. Classical data integration approaches need a global schema before queries on the integrated data can be stated. Dataspace is a new abstraction of data management, that doesn't enforce any schema, and doesn't enforce full data integration in query results. Thus a dataspace allows e.g. that semantical equal data is not presented equal. This also called the acceptance of \emph{data co-existence}. A dataspace also needs only low upfront work for datasources thus being suitable for highly heterogeneous environments. In this thesis we present MeDSpace, a distributed system that allows to state keyword search queries over heterogeneous datasources. The presented system is a test environment for medical datasources and can be used as a starting point for creating a dataspace over multimedia datasources. The system uses medical data, but could be used for any kind of multimedia data that should be searchable by keywords.