\chapter{Conclusion}

In this thesis we presented MeDSpace, a testsuite for medical data. It allows to execute a distributed keyword search on heterogeneous datasources, containing medical information. 
Additionally, this system provides wrappers for a relational datasource that uses SQL for querying, a DDSM image file server and a PDF file server. The wrappers are able to register and deregister themselves from the MeDSpace system by calling the corresponding register and deregister REST services. MeDSpace supports query and search result caching, keyword search using AND, or optional OR operators, and query results can be displayed directly in the browser or downloaded as files. Additionally the system has an io-error policy for datasources, so that dead datasources are removed automatically. Search results of images and pdf files contain URLs to the source files. The wrappers for the respective datasource provide a service so that the source files can be downloaded.

The system was designed to be used as a test environment for implementing dataspace features. It can also be used as a starting point for a much more evolved dataspace.

There are many fields that could be improved: the system allows to cache queries. This could be used as a starting point for implementing incremental (\emph{pay-as-you-go}) data integration which would move MeDSpace more into the direction to be an actual dataspace. Furthermore, MeDSpace uses RDF as its canonical data model. Therefore an ontology-based data integration service could be implemented that uses ontologies specified in OWL. 
Currently MeDSpace also uses  a very primitive method to do the keyword search query.
A possible improvement here would be to implement a global query language that considers also multimedia data. A starting point could be this work \cite{6214725}.