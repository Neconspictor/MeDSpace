Dataspaces describes a new abstraction of data management \cite{Franklin:2005:DDN:1107499.1107502}. In current scenarios it is rarely the case, that data to be managed is solely stored in a convenient relational Database Management System (DBMS) or in another, single data model. Therefore, developers often face the challenge to deal with heterogeneous data on a low level base. Challenges are to provide search and query capabilities, enforcing rules, integrity constraints, naming conventions etc.; tracking lineage; providing availability, recovery and access control; and managing evolution of data and meta-data. That issues are ubiquitous and arise in enterprises, government agencies and even on one's PC desktop. 

As a response to this problem, the authors postulate the concept of dataspaces and corresponding to this, the  development of DataSpace Support Platforms (DSSPs). Shortly said, the latter provides an environment of cooperating Services and guaranties, that enables software developers to concentrate on their specific application problem rather than taking care of returning issues in consistency and efficiency of huge, linked but heterogeneous data sets. The remarkable properties of a dataspace system are defined as follows:

A DSSP must deal with data and applications in a variety of file formats, that are accessible through many systems with different interfaces. A DSSP has to support all kinds of data in the dataspace rather than only a few (as DBMSs do).
Although a DSSP provides a integrated possibility for searching, querying, updating and administration, the data often can only be accessible and modifiable through native interfaces. Therefore DSSPs haven't full control over their data.
Queries on a DSSP may offer varying levels of services. In some cases the answers can be approximated resp. best-effort. An example: If some data sources are unavailable for some reasons the DSSP is able to return the best result as possible. Therefore it uses the data that are available at the time of the query.
A DSSP has to provide the tools that allow a tighter data integration process in the dataspace as necessary.

Many of the services a dataspace provide, data integration and exchange systems provide, too. The main difference between these systems is that data integration systems need a semantic integration process before they can provide any services on the data. But dataspace is not kind of a classic data integration approach. In order to avoid semantic integration, a dataspace uses the concept of data coexistence.  The idea is to provide base functionality over all data sources, regardless of their specific integration constraints. E.g. a DSSP is able to provide a keyword search similar to a desktop file search. If more sophisticated operations are required such as relational queries, data mining or monitoring of specific data sources, additional effort can be done to integrate the sources tighter. This incremental process is also called as ``pay-as-you-go'' fashion.

In chapter 3 of the paper the authors pottered at the logical components and services a DSSP should have:

\textbf{Logical components}

A Dataspace should contain all information being relevant for a specific organization/task regardless of their file format or storage location, and it should model a collection of relationships between the data repositories. Therefore the authors define the dataspace as a set of participants and relationships. Participants of a dataspace are individual data sources . Some participants support expressive query languages for querying while others only provide limited access interfaces. Participants can reach from structured, semi-structured right up to unstructured data sources. Some sources provide traditional updates, some are only be appendable, while others are immutable. Further, dataspaces can be nested within each other, which means that a dataspace should be able to be a part of another dataspace.  Thus, a dataspace has to provide methods and rules for accessing its sub dataspaces.

\textbf{Services of dataspaces}

The most basic services a dataspace should contain is the cataloging of data elements of all participants. A catalog is an inventory of data resources containing all important information about every element (source, name, storage location inside the source, size, creation date, owner, etc.)of the dataspace. It is the infrastructure for the most other dataspace services. Search and query are two main services a DSSP must provide. A user should be able to state a search query and iteratively precise it, when appropriate, to a database-style query. For the dataspace approach it is a key tenet that search should be applicable to all of the contents, regardless of their formats.
The search should include both data and meta-data. The user should be enabled to discover relevant data sources and inquire about the completeness, correctness and freshness. A DSSP in fact should be aware of the gaps in its coverage of the domain.  
A DSSP should also support updating data. Of course, the mutability of the relevant data sources determines the effects of updates.  
Other key services would be monitoring, event detection and the support for complex workflows (e.g. it is desired that a calculation is done if new data arrives and that the result will be distributed over a set of data sources). On a similar way a DSSP should support various forms of data mining and analysis. 
Not every participant will provide all necessary interfaces for being able to support all DSSP features. Hence it is necessary, that data sources can be extended on various ways. E.g. a source don't store its own meta-data, so there has to be an external meta-data repository for it. 

\textbf{Components and architecture of a dataspace system}

\uline{Catalog and Browse} Information about all participants and the relationships among them are stored in the catalog. It must also deal with a large variety of sources and supports to provide different levels of information about their structure and capabilities. It is important, that the catalog includes the schema of the source, statistics, rates of change, accuracy, completeness query answering capabilities, ownership, and access and privacy policies for each participant. Relationships may be stored as query transformations, dependency graphs or even textual descriptions. If possible, the catalog should include a basic inventory of the data elements at each participant: identifier, type, creation data and so forth. Then, it can support basic browsing over the participant's inventories.
It isn't a very scalable interface, but it can be used to response to questions about the presence or absence of a data element, or determine which participants have documents of a specific type. 
On top of the catalog, the DSSP should have a model-management environment allowing the creation of new relationships and manipulation of existing ones.

\uline{Search and Query} The component has to offer the following capabilities:

(1) \emph{Query everything:} Any data item should be queryable  by the user regardless of the file format or data model. Keyword queries should be supported, initially. When more information about a participant is collected, it should be possible to gradually support more sophisticated queries. The transition between keyword query, browsing and structured querying should be gracefully. And when answers are given to keyword (or structured) queries, the user should be able to refine the query through additional query interfaces. 

(2)\emph{Structured query:} Queries similar to database ones should be supported on common interfaces (i.e. mediated schemas) that provide access to multiple sources or can be posed on a specific data source (using its own schema). The intention is, that answers will be obtained from other sources (as in peer-data management systems), too. Queries can be posed in a variety of languages (and underlying data models) and should be translated into other data models and schemas as best as possible with the use of exact and approximate semantic mappings.

(3) \emph{Meta-data queries:} It is essential, that the system supports a huge variety of meta-data queries. These include (a) source inclusion of an answer or how it was derived or computed, (b) timesteps provision of the data items that are included in the computation of an answer, (c) specification of whether other data items may depend on on a particular data item and the ability to support hypothetical queries. A hypothetical question would be 'What would change if I removed data item X?. (d) Querying the sources and degree of uncertainty about the answers.
Queries locating data, where the answers are data sources rather than specific data items, should be supported, too. 

(4) \emph{Monitoring:} All stated Search and Query services should also be supported in an incremental form which is also applicable in real-time to streaming or modified data sources. It can be done either as a stateless process, in which data items are considered individually, or as a statefull process. In the latter multiple data items are considered.   

\uline{Local store and index} A DSSP has a store and index component to achieve the following goals:
to create efficiently queryable associations between data items in different participants. Important is here, that the inde should identify information across participants when certain tokens appear in multiple ones (in a sense, a generalization of a join index)
to improve accesses to data items with limited access patterns. Here, the index has to be robust in the face of multiple references to real-world objects, e.g. different ways to refer to a company or person.
to answer certain queries without accessing actual data sources. Thus the query load is reduced on participants which cannot allow ad-hoc external queries. 
to support high availability and recovery

The index has to be highly adaptive to heterogeneous environments. It should take as input any token appearing in the dataspace and return the location at which the token appears and the roles at each occurrence. Occurrences could be a string in a text file, element in file path, a value in a database, element in a schema or tag in a XML file. 

\uline{Discovery Component} This components locates participants in a dataspace, creates relationships between them, and helps administrators to refine and tighten these relationships. For each participant the component should perform an initial classification according to the participant's type and content. The system should provide an environment for semi-automatically creating relationships between existing participants and refining and maintaining existing ones. This involves both finding which pairs of participants are likely to be related, and then proposing relationships which a human can verify and refine. The discovery component should also monitor the content in order to propose additional relationships in the dataspace over time.  
 
\uline{Source Extension Component} Some participants may don't provide significant data management functions. For example, a participant might be no more than departmental document repository, perhaps with no services than weekly backups. A DSSP should support to enrich such a participant with additional capabilities, such as a schema, a catalog, keyword search and update monitoring. It may be necessary to provide these extensions locally as there can be existing applications or workflows that assume the current formats or directory structures.	
This component also supports ``value-added'' - information held by the DSSP, but not present in the initial participants. Such information can include ``lexical crosswalks'' between vocabularies, translation tables for coded values, classifications and ratings of documents, and annotations or linked attached data set or document contents. Such information must be able to span participants in order to link related data items.