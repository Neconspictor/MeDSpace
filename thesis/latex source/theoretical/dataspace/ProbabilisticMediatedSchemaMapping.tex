\subsection{Probabilistic Mediated Schema Mapping}
\textcolor{red}{\textbf{TODO}}\\
\textcolor{red}{Used Papers: \cite{DasSarma:2008:BPD:1376616.1376702}}\\

In \cite{DasSarma:2008:BPD:1376616.1376702} was shown, that it is possible to automatically bootstrap a data integration system that answers queries with high precision and recall. In a sense, it is possible to set up a data integration system without any human involvement. With this approach, it isn't possible to answer queries fully accurate and complete as it would be with manual or semi-automatic ones, but best-effort answers and improvement over time are possible by using a probabilistic data model. 
Mediated schemas can be build automatically by clustering attributes from the various source schemas to groups by their semantic meaning and then using this groups as attributes in the mediated schema. Sources in a dataspace typically are heterogeneous, so as a general rule there is more than one possibility to cluster the attributes and it exists an uncertainty of the best(s) grouping(s). If you would choose only one schema of the set of possible mediated schemas, you would likely have a loss in query answering in both terms, precision and recall.  So it seems natural to consider all possible mediated schemas combined in a single mediated schema. By weighting each mediated schema through its likelihood of accuracy, it is possible to favor schemas with a greater weight for query answering and answer ranking. And that is exactly, what a probabilistic mediated schema (p-mediated schema) is doing.
For clustering attributes from the source schemas, similarity functions based on attribute matching are used. Thus, some not very obvious attribute correspondences aren't detected by this approach. But the authors believe, that using more advanced schema matching algorithms like also considering column value similarities, could minimize this problem \cite{DasSarma:2008:BPD:1376616.1376702}.

The concept of probabilistic Schema Mapping (p-mapping) was introduced in \textcolor{red}{Cite!!!}. It describes a probabilistic distribution of possible mappings between a source and a mediated schema. A mathematical formal definition and additional information about probabilistic schema mapping can be found in the original paper \cite{DasSarma:2008:BPD:1376616.1376702}.
 
The whole process is visually represented in \textcolor{red}{\textbf{\{Reference figure for p-mediated data integration system!\}}}, which shows the architecture of the data integration system used by \cite{DasSarma:2008:BPD:1376616.1376702}.
On set-up time, the system automatically generates the p-mediated schema and appertaining p-mappings and consolidates them to generate a final mediated schema and mappings. Consolidation means to generate one single mediated schema. The purpose of the consolidation is, that the user has a sole mediated schema to interact with and additionally it speeds up query answering, as queries only need to be rewritten and answered based on only one mediated schema. At query-answering time the query is rewritten for each data source according to the mappings and answer the rewritten query on the data sources. Answering queries with respect to p-mappings returns a set of answer tuples, each with a probability indicating the likelihood that the tuple occurs as an answer.

\textcolor{red}{\textbf{//--------------------------------------------------------------------------------\\}}\\