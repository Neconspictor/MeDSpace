\section{Multimedia Dataspaces}

%In the paper 'Towards a Model for Multimedia Dataspaces'\cite{6167826} the authors present a representation model for dataspaces that overcomes the shortcomings of existing dataspace models.

Research in the field of dataspaces is currently very active, but despite of the deep interest, existing models suffer on a number of shortcomings limiting their applicability \cite[p. 1]{6167826}.
These include the tendency of overlooking the different types of relations that can exist between data items, which restricts the amount of information they can generally integrate.
Second, they focus on classical text data ignoring the specifics of multimedia data. Third they don't provide the fine-granularity in the persistence of integrated data that is needed in certain domains.

To address these issues, a representation model for dataspaces was developed, which sees the dataspace as a set of classes, objects (instances of classes) and relations \cite[p. 1]{6167826}.
In the latter case there exist relations between classes (CRC), relations between objects and classes (ORC) and relations between objects (ORO).
Furthermore relations can be internal or external defining whether the anticipating relation objects are in the same data source or in different ones.

A design goal of the model is the maximization of its expression in terms of the types of relations that it can represent, enabling it to deal with information originating from structured data, semi-structured data, ontologies and other forms of knowledge representation as well as from canonical  
knowledge.

Important to note is the fact that the model includes \emph{similarity relations} in the type of relations \cite[p. 3-4]{6167826}. This property makes this model also suitable for integrating multimedia data.
Similarity functions are a feature of multimedia data that defines a measurement for non-exact matching between objects.
This kind of relations can be used to derive relations between other objects in the dataspace.

Additionally the model introduces the concept of \emph{dataspace-views} \cite[p. 3]{6167826}. 
This makes it possible to store query results of an existing dataspace into a new sub-dataspace with different modes of persistence(virtualized view, materialized view, mode of synchronization with the content of the original sources,...).

The multimedia dataspace model was implemented in \cite[p. 30]{ZerrikBachelorThesis} using RDF as its canonical data model.


\textcolor{gray}{Nice to have: Example image demonstrating the multimedia data model}