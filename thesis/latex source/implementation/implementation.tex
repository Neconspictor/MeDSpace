\chapter{Implementation}

\textbf{Importance of unicode}

A dataspace includes data sources possibly spread around the globe. It is near inferring to support a wide area of different languages. In terms of character sets, it is therefore necessary to use a unicode encoding.
Unicode is a system assigning each character a unique code point and it is designed to support the worldwide interchange, processing and display of texts written in different languages\cite{UnicodeStandard}.\newline
There exist several encodings for unicode. The more well known are the UTF and UCS encoding families. In the draft of HTML5 it is advised to use UTF-8 for new web pages\cite{HTML5Rec}. 
Thus, to simplify processing, we follow the recommendation and use UTF-8 throughout the dataspace. If a data source doesn't use UTF-8, it is the task of its wrapper to do a proper conversion to UTF-8.

\section{Setting up databases}


\subsection{Setting up a MySQL data source}

To setup a MySQL data source download a recent stable MySQL community server\footnote{\url{https://www.mysql.de/downloads/}} 
and install it for you target platform. Additionally you will need the Connector/J components, the official JDBC driver for MySQL. 

At time of writing, the most recent stable versions are the community server 5.7.17 and the Connector/J 5.1.41. These versions are used for the thesis project and all following commands are related on them. If you're using 
different versions, assure that the instructions are adapted properly. As a detailed installation instruction for all supported platforms would break the mold, the reader is encouraged to consult the official 
manual\footnote{\url{https://dev.mysql.com/doc/}}. 
Assure that the MySQL binary folder is integrated into your class path, so that you can access it globally in a shell/command line. Although not necessary it is recommended for security reasons 
to set a password for the root user\footnote{\url{https://dev.mysql.com/doc/refman/5.7/en/default-privileges.html} , \url{https://dev.mysql.com/doc/refman/5.7/en/resetting-permissions.html}}. 
After installing the server, do postinstallation setup and testing\footnote{\url{https://dev.mysql.com/doc/refman/5.7/en/postinstallation.html}}. 

To support UTF-8, set in your my.cnf 

\begin{codebox}
	default-character-set = utf8
\end{codebox}

in the mysql section and

\begin{codebox}
	character-set-server=utf8\newline
	collation\-server=utf8\_general\_ci
\end{codebox}

in the mysqld section. Then restart the mysqld daemon. In the following, it is assumed, that you have a running MySQL server now that can be accessed via shell/command line. Before you connect to the MySQL server, you should assure that the application you use for connecting is using UTF-8 for user input and sending statements. So, validate that your shell/command line is using UTF-8. E.g. on windows system the command line isn't using UTF-8 by default
\footnote{To set the encoding to UTF-8 on the windows command line, change the active code page to 65001 and set 'Lucida Console' as the displaying font. In contrast to the font, the code page is only active for the current console session. But you can automate this command with a AutoRun setting. For more information see \url{https://blogs.msdn.microsoft.com/oldnewthing/20071121-00/?p=24433}}.  
Now try to connect to the database as the user root:

\begin{codebox}
	mysql -u root -p 
\end{codebox}

If you've done all right, you should be connected to the database after entering and confirming the password, that you've previously stated for the user root.

The next step is to validate, that UTF-8 is indeed continuously used. Execute:

\begin{codebox}
	SHOW VARIABLES LIKE 'char\%';
\end{codebox}

and check, that the variables \emph{character\_set\_client}, \emph{character\_set\_connection}, \emph{character\_set\_database}, \emph{character\_set\_results}, \emph{character\_set\_server} and \emph{character\_set\_system} are all set to utf8.
Basically, these variables are used to interpret and write data consistently in UTF-8. More information about the stated variables can be found on the manual
\footnote{\url{https://dev.mysql.com/doc/refman/5.7/en/server-system-variables.html\#sysvar_character_set_client}}.

The next step is to initialize the data source with a database and some content. Further we need a user which is used by the wrapper to communicate with the data source. The wrapper needs no writing rights and indeed we don't want it to change the data, so following the security rule 'As few rights as possible' we grant that user only reading rights for fetching data. The commands for initializing the data source and creating a read-user are in the file \textbf{init.sql} which is located in the appendix data in the folder \textbf{implementation/SQL}.

To execute commands from a file, execute while logged in as the root user:

\begin{codebox}
	SOURCE \emph{path\_to\_sql\_file};
\end{codebox}

where \emph{path\_to\_sql\_file} is the full (absolute or relative) path to the sql file.
Per default, the created database will be named \textbf{medspace} and the read-user will be called \textbf{medspace\_client}. If you want edit the init.sql file, beware that the file is encoded in UTF-8. As we instructed mysql to use UTF-8 in every case, this is encoding is required. Assure that your file editor saves the file in that encoding, too.


\section{Wrappers}
\section{Register}
\section{Connecting PHP server to Java backend}

I decided to use both, PHP and Java in back end. 

\section{Basic dataspace querying}
\section{Advanced query techniques/features}
\section{GUI}
\section{Validation}