\chapter*{Abstract (german translation)}

Der Gesundheitsbereich ist mit dem Problem konfrontiert, dass eine gewaltige Menge von heterogenen Daten existiert, die nicht einfach so abrufbar sind. Es wäre daher hilfreich ein System zu haben, dass dieses Problem schmälert. Klassische Informationsintegration Lösungen brauchen ein globales Schema bevor Anfragen an die integrierten Daten gestellt werden können. Auch wenn solche Ansätze funktionieren, sind sie doch sehr schwerfällig und können nicht dynamisch an die sich ständigenden Veränderungen im Gesundheitsbereich angepasst werden ohne erheblichen Verwaltungsaufwand. Dataspace ist eine neue Abstraktion für Datenmanagement, das kein globales Schema zwingend benötigt und die Koexistenz von Daten im Integrationssystem erlaubt. Zusätzlich wird ein verhältnismäßig geringer Arbeitsaufwand benötigt, um Basisdienste in einem Dataspace anbieten zu können. Das macht ein Dataspace attraktiv für besonders heterogene Umgebungen wie für den Gesundheitsbereich. 

In dieser Arbeit präsentieren wir MeDSpace, ein verteiltes System, dass es ermöglicht Stichwortsuchanfragen an eine Menge von heterogenen Datenquellen zu stellen. Das System ist eine Testumgebung für medizinische Datenquellen und kann als Startpunkt verwendet werden, um ein Dataspace über medizinische Datenquellen zu entwickeln. Auch wenn das System medizinische Datenquellen verwendet, kann es trotzdem für jegliche Art von Multimediadaten verwendet werden, die über eine Stichwortsuche verfügbar sein sollen.