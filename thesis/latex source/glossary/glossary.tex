\chapter{Glossary}

\textbf{Data model} In data modeling theory, a data model is an abstract model that describes how data is represented and used. It consists of a set of data structures and conceptual tools that is used to describe the structure of a database and operations that can be performed on the data. A data model consists of a data model theory, a formal description of the data model, and a data model instance, which is a practical data model designed for a particular application. There exist three types of data models \cite[p.10-14]{IntroductionToDatabaseSystems2010}:
\begin{itemize}
	\item \textbf{Conceptual data model}: Describes data independent from any implementation details. It is used to describe the semantic of a domain. Examples for conceptual data models are the entity-relationship (E-R) model and ontologies. 
	\item \textbf{Logical data model}: Also called representational or implementation data model. Describes data in terms of data structures like relational tables and columns or XML tags. Examples of logical data models are the relational data model, the object-based data model, semi-structured data models like XML or a graph-based data model like RDF. 
	\item \textbf{Physical data model}: Describes data in terms of collection of files, indices and other physical storage structures. It defines how the data is stored on disk and what access methods are available to it. 
\end{itemize}

\textbf{Data source} A data source addresses an arbitrary data storage, whose data should be integrated in an information system. (\cite[p. 7]{DBLP:books/dp/LeserN2006}, own translation)

\textbf{Integrated or integrating information system} An integrated information system is an application, which facilitates the access to different data sources (\cite[p. 7]{DBLP:books/dp/LeserN2006}, own translation)

\textbf{Meta data} Data that describes other data. The distinction between data and meta data in a system, is dependent on the particular application. Meta data has to be stored, browsed and integrated as well as 'normal' data. (\cite[p. 8]{DBLP:books/dp/LeserN2006}, own translation)

\textbf{Schema} The word comes from greek \foreignlanguage{greek}{\emph{σχήμα}}(skhēma) meaning \emph{shape} or \emph{plan}. In computer science the word has different meanings. 
In the context of database theory the word is used for a representation of the structure (syntax), semantics, and constraints on the use of a database (or its portion) in a particular data model. \cite[p. 235]{Sheth:1990:FDS:96602.96604}. Analogous, an XML schema defines the structure, content and semantics of XML documents\cite{w3XMLSchema}. To generalize it, a schema in the context of data integration is the definition of the structure, content and semantics of a data source.


